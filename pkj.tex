\documentclass{article}
\usepackage{amsmath, amssymb}
\usepackage{geometry}
\usepackage{graphicx}

\geometry{margin=1in}

\newcommand{\pointA}{(1, -1)}
\newcommand{\pointB}{(-4, 6)}
\newcommand{\pointC}{(-3, -5)}
\newcommand{\pointO}{\left(-\frac{53}{12}, \frac{5}{12}\right)}

\begin{document}
\title{
\Huge \ Assignment 1\\
\Huge\ Probability And Random Processes\\
\large\author{Pankaj Kumar \\EE22BTECH11040}
}
\maketitle

Given:
\[
\vec{A}\pointA, \vec{B}\pointB, \vec{C}\pointC, \quad 
\]
And we find point O which is point of intersection of perpendicular bisector of AB and AC;\\
\[\vec{O} = \left(-\frac{53}{12}, \frac{5}{12}\right)\\\]


Verify that : $OA = OB = OC$\\

\textbf{Proof:}\\


The vectors $A$, $B$, $C$ and $O$ are:
\[
\begin{aligned}
\vec{A} = (1, -1), \\
\vec{B} = (-4, 6), \\
\vec{C} = (-3, -5)\\
\vec{O} = \left(-\frac{53}{12}, \frac{5}{12}\right)
\end{aligned}
\]

Using the distance formula $\sqrt{(x_2 - x_1)^2 + (y_2 - y_1)^2}$, we can calculate the distances:\\

\[
\begin{aligned}
OA &= \|\vec{OA}\| = \sqrt{(1 - (-53/12))^2 + (-1 - 5/12)^2}  =  5.59, \\
OB &= \|\vec{OB}\| = \sqrt{(-4 - (-53/12))^2 + (6 - 5/12)^2}  =  5.59, \\
OC &= \|\vec{OC}\| = \sqrt{(-3 - (-53/12))^2 + (-5 - 5/12)^2}  =  5.59.
\end{aligned}
\]
From above we can see that :\\
\[OA = OB = OC\]
\large Hence, proved

\end{document}
